In lab 4 we wrote a testbench for our processor. The testbench was responsible
for driving the inputs to the processor (\textbf{CLK}, \textbf{ARESETN} and
\textbf{master\_load\_enable}), as well as checking the correctnes of the
outputs from the processor. The testbench ran a simple example program on the
processor, and was given the correct outputs for this program. The program code
is given in figure \ref{code:lab4}. 

\begin{figure}
    \caption{Example program in Chacc-assembly for lab 4.}
    \label{code:lab4}
    \centering
    \begin{lstlisting}[language=c]
    Load Byte 00000001  // ACC = DM[1]
    Add 00000000        // ACC = ACC + DM[0]
    Display             // Move ACC to Display Reg
    Store byte 00000001 // DM[1] = acc
    subtract 00000000   // ACC = ACC - DM[0]
    Store 00000000      // DM[0] = ACC;
    Jump 00000000       // start over
    \end{lstlisting}
\end{figure}

\subsection*{Driving the inputs}
Driving the inputs was a
straightforward process. We begun by driving the clock (\textbf{CLK})
switching its value between 1 and 0 every 5ns. Giving the clock a period of
10ns. We then added a process with \textbf{CLK} as its only dependency to drive
the other two inputs. In this process, the master load enable signal is toggled
every time the \textbf{CLK} is changed. On every rising edge of \textbf{CLK} the
process increments a counter \textbf{CLK\_step}, when this counter is below 2
the reset signal is active (\textbf{ARESETN} = 0), when the counter is 2 or
higher the reset signal is inactive (\textbf{ARESETN} = 1).
\subsection*{Controlling the outputs}
The correct outptus for the example program in figure \ref{code:lab4} were given
in a series of files for the different output signals. For each output we made
an array of values which where filled by reading these files. Then, for each
output a process was created, so that each time that particular output changed
an assertion was run checking it for its corresponding value in the array. The
VHDL for one such process is given in figure \ref{code:lab4Process}. To
determine when we were done we stopped the program when the \textbf{disp2seg}
signal showed $90_{16}$
\begin{figure}
    \caption{VHDL process for checking the values of acc2seg. }
    \label{code:lab4Process}
    \centering
    \begin{lstlisting}[language=vhdl]
    process(acc2seg)
    begin
    if (ARESETN = '1') then
      assert (acctrace(acc_step) = acc2seg)
        report "acc is not what it should be.";
      acc_step <= acc_step + 1;
    end if;
    end process;
    \end{lstlisting}
\end{figure}